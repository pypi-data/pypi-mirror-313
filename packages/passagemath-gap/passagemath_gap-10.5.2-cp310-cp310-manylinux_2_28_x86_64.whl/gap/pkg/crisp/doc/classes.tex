%!TEX root = manual.tex
% !TeX TS-program = pdftex
% !TEX spellcheck =en_GB
%%%%%%%%%%%%%%%%%%%%%%%%%%%%%%%%%%%%%%%%%%%%%%%%%%%%%%%%%%%%%%%%%%%%%%%%%%%%%
%%
%%  classes.tex              CRISP documentation            Burkhard Höfling
%%
%%  Copyright © 2000, 2011 Burkhard Höfling
%%
%%%%%%%%%%%%%%%%%%%%%%%%%%%%%%%%%%%%%%%%%%%%%%%%%%%%%%%%%%%%%%%%%%%%%%%%%%%%%
\Chapter{Set theoretical classes}

In {\CRISP}, a class (in the set-theoretical sense) is usually represented
by an algorithm which decides membership in that class. Wherever this makes
sense, sets (see "ref:Set") may also be used as classes.


%%%%%%%%%%%%%%%%%%%%%%%%%%%%%%%%%%%%%%%%%%%%%%%%%%%%%%%%%%%%%%%%%%%%%%%%%%%%%
\Section{Creating set theoretical classes}\null

\index{classes!creation of}

\>IsClass(<C>) C

returns true if <C> is a class object. The category of class objects is a
subcategory of the category `IsListOrCollection'.

\>Class(<rec>) O
\>Class(<func>) O

returns a class <C>. In the first form, <rec> must be a record having a
component `\\in' and an optional component `name'. The values of these
components, if present, are bound to the attributes `MemberFunction' and
`Name' (see "ref:Name") of the class created. The value bound to `\\in' must be a function
<func> which returns `true' if a {\GAP} object belongs to <C>, and `false'
otherwise; cf. "MemberFunction" below. The second form is equivalent to `Class(rec(\\in
:= <func>))'. It is the user's responsibility to ensure that <func> returns the same
result for different {\GAP} objects representing the same mathematical object (or
element, in the {\GAP} sense; see "ref:Objects and Elements" in the {\GAP}
reference manual).

\beginexample
gap> FermatPrimes := Class(p -> IsPrime(p) and p = 2^LogInt(p, 2) + 1); 
Class(in:=function( p ) ... end)
\endexample

\>View(<class>)!{for classes}

If the class does not have a name, this produces a brief description of the
information defining <class> which has been supplied by the user. If the
class has a name, only its name will be printed.

\begintt
gap> View(FermatPrimes);
Class(in:=function( p ) ... end)
\endtt

\>Print(<class>)!{for classes}

`Print' behaves very similarly to `View', except that the defining
information is being printed in a more explicit way if possible.

\begintt
gap> Print(FermatPrimes);
Class(rec( in = function( p )
    return IsPrime( p ) and p = 2 ^ LogInt( p, 2 ) + 1;
end))
\endtt


\>Display(<class>)!{for classes}

For classes, `Display' works exactly as `Print'.


\>`<obj> in <class>'{element test!for classes}

\indextt{in!for classes}\relax
\index{membership test!for classes}\relax
returns true or false, depending upon whether <obj> belongs to <class> or
not. If <obj> can store attributes, the outcome of the membership test is
stored in an attribute `ComputedIsMembers' of <obj>.

\>`<C1> = <C2>'{equality!for classes}

Since it is not possible to compare classes given by membership algorithms,
two classes are equal in {\GAP} if and only if they are the same {\GAP}
object (see "ref:IsIdenticalObj" in the {\GAP} reference manual).

\>`<C1> \<\ <C2>'{comparison!for classes}

The operation `\<' for classes has no mathematical meaning; it only exists
so that one can form sorted lists of classes.


%%%%%%%%%%%%%%%%%%%%%%%%%%%%%%%%%%%%%%%%%%%%%%%%%%%%%%%%%%%%%%%%%%%%%%%%%%%%%
\Section{Properties of classes}\null

\index{classes!properties of}
\index{properties!of classes}

\>IsEmpty(<C>)!{for classes} P

This property may be set to `true' or `false' if the class <C> is empty
resp. not empty.

\>MemberFunction(<C>) A

This attribute, if bound, stores a function with one argument, <obj>,
which decides if <obj> belongs to <C> or not, and returns `true' and `false' accordingly.
If present, this function is called by the default method for `\\in'.
`MemberFunction' is part of the definition of <C> and should not be called 
directly by the user.



%%%%%%%%%%%%%%%%%%%%%%%%%%%%%%%%%%%%%%%%%%%%%%%%%%%%%%%%%%%%%%%%%%%%%%%%%%%%%
\Section{Lattice operations for classes}\null

\index{lattice operations!for classes}

\>Complement(<C>) O

returns the unary complement of the class <C>, that is, the class consisting
of all objects not in <C>. <C> may also be a set.

\beginexample
gap> cmpl := Complement([1,2]);
Complement([ 1, 2 ])
gap> Complement(cmpl);
[ 1, 2 ]
\endexample

\>Intersection(<list>)!{of classes} F
\>Intersection(<C1>, <C2>, \dots)!{of classes} F
\indextt{INTERSECTION\noexpand_LIMIT}\relax

returns the intersection of the groups in <list>, resp. of the classes
<C1>, <C2>, \dots. If one of the classes is a list with fewer than
`INTERSECTION_LIMIT' elements, then the result will be
a sublist of that list. By default, `INTERSECTION_LIMIT' is 1000. 

\beginexample
gap> Intersection(Class(IsPrimeInt), [1..10]);
[ 2, 3, 5, 7 ]
gap> Intersection(Class(IsPrimeInt), Class(n -> n = 2^LogInt(n+1, 2) - 1));
Intersection([ Class(in:=function( N ) ... end), 
  Class(in:=function( n ) ... end) ])
\endexample

\>Union(<C>, <D>) F

returns the union of <C> and <D>.  

\beginexample
gap> Union(Class(n -> n mod 2 = 0), Class(n -> n mod 3 = 0));
Union([ Class(in:=function( n ) ... end), Class(in:=function( n ) ... end) 
 ])
\endexample

\>Difference(<C>, <D>) O

returns the difference of <C> and <D>. If <C> is a list, then the
result will be a sublist of <C>.

\beginexample
gap> Difference(Class(IsPrimePowerInt), Class(IsPrimeInt));
Intersection([ Class(in:=function( n ) ... end), 
  Complement(Class(in:=function( N ) ... end)) ])
gap> Difference([1..10], Class(IsPrimeInt));
[ 1, 4, 6, 8, 9, 10 ]
\endexample

%%%%%%%%%%%%%%%%%%%%%%%%%%%%%%%%%%%%%%%%%%%%%%%%%%%%%%%%%%%%%%%%%%%%%%%%%%%%%
%%
%E
%%

